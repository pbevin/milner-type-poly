\documentclass[12pt]{article}
\usepackage[margin=1in]{geometry}
\usepackage{amsmath,amsthm,amssymb,amsfonts,mathrsfs,txfonts,booktabs,enumitem,lmodern}

\setlist{nosep}

\newcommand{\B}{\mathbb{B}}
\newcommand{\N}{\mathbb{N}}
\newcommand{\R}{\mathbb{R}}
\newcommand{\instanceof}[2]{#1\textsf{\textbf{E}}#2}
\newcommand{\cast}[2]{#1\mid#2}
\newcommand{\pe}[2]{#1\mid#2}
\newcommand{\tpe}[2]{\overline{#1}\mid\overline{#2}}
\newcommand{\eval}[1]{\mathscr{E}[\![#1]\!]\eta}
\newcommand{\env}[1]{\eta[\![#1]\!]}

\begin{document}

%\renewcommand{\qedsymbol}{\filledbox}
%Good resources for looking up how to do stuff:
%Binary operators: http://www.access2science.com/latex/Binary.html
%General help: http://en.wikibooks.org/wiki/LaTeX/Mathematics
%Or just google stuff

\title{A Theory of Type Polymorphism in Programming: The~Cheat Sheet}
\author{Pete Bevin}
\date{April 2015}
\maketitle

\section{Overview of the Paper}
\begin{enumerate}
\item Introduction
\item Illustrations of the Type Discipline
\item \textbf{A Simple Applicative Language and its Types}
  \begin{itemize}
  \item The \texttt{exp} language, its semantics, and what type errors it can raise at runtime (3.1, 3.2)
  \item Types in general (3.3, 3.4)
  \item What it means to be well-typed (3.5)
  \item Substitutions (3.6)
  \item The Semantic Soundness theorem (if we can assign a type, the runtime won't raise errors) (3.7)
  \end{itemize}
\item \textbf{A Well-Typing Algorithm and its Correctness}
  \begin{itemize}
  \item Algorithm W
  \item The Syntactic Soundness theorem
  \item Algorithm J
  \end{itemize}
\item Types in Extended Languages
  \begin{itemize}
  \item Tuples, union types, and lists
  \item Assignable variables and assignments
  \item Recursive type declarations
  \end{itemize}
\end{enumerate}

\section{Symbols}

\begin{tabular}{l p{.25\textwidth} p{.5\textwidth}}
\toprule
Term & Definition in Paper & Meaning \\
\midrule
$B_0$ & $T = \{\text{true}, \text{false}, \bot_T\}$ & Boolean types (including $\bot$) \\
$B_1$, \ldots, $B_n$ & Other types & Integers, reals, strings, pairs of types, etc. \\
$V$ &  & \\
$W$ & Wrong & The error type \\
$\eta$ & Environment & In the paper, this is a function from \texttt{id} to \texttt{V}, but we would be more likely to model it as a \texttt{map}.\\
$\mathscr{E}$   & Semantic Function & \texttt{eval} \\
$\eval{T}$ & Semantic evaluation & \texttt{(eval exp env)} where \texttt{eval} is $\mathscr{E}$, \texttt{exp} is $T$, and \texttt{env} is $\eta$ \\
$\env{x}$ & Type lookup & The type of the Exp variable $x$. \\
$\iota_n$ & Type of $B_n$ & For example, $B_0$ is $\B$, $B_1$ might be $\N$, $B_2$ might be $\R$, etc. \\
$\instanceof{v}{D}$ & [16] page~34 & Value $v$ has type $D$ - like \texttt{instanceof} in Java. \\
$\cast{v}{D}$ & & Type cast (error value if not possible, but we always check first with $\instanceof{v}{D}$) \\

$\pe{p}{e}$ & Prefixed expression & An expression showing what $\lambda$, \texttt{fix}, and \texttt{let} bindings are in effect \\
$\tpe{p}{e}$ & Typed expression &  A prefixed expression augmented with type information at each level \\
$\sqsubseteq$ & No definition given & Measure of ``defined''ness -- e.g., in $B_0$, $\bot_T \sqsubseteq \text{true}$ and $\bot_T \sqsubseteq \text{false}$, but $\text{false} \nsqsubseteq \text{true}$. \\
$\sqcup$ & No definition given & \\

\bottomrule
\end{tabular}



\section{Concepts}

\begin{tabular}{l l p{.6\textwidth}}
\toprule
Page & Concept & Meaning \\
\midrule
361 & Prefix & A list of the variable bindings in effect for an expression \\
361 & Active & A member of the prefix that is not shadowed by a later member with the same name \\
362 & Generic type variable & One that does not occur in the type of any $\lambda$ or \texttt{fix} binding above it \\
362 & Standard typing & You can safely ignore this: it's a purely technical constraint. \\
362 & Well-typed & An expression is well-typed if it can be assigned a type and the resulting type obeys certain laws. See Proposition~3 on page~362.\\
364 & Semantic Soundness Theorem & A well-typed program cannot ``go wrong'' \\
367 & Syntactic Soundness Theorem & If algorithm $\mathscr{W}$ accepts a program, then it is well-typed. \\

\bottomrule
\end{tabular}


\end{document}
